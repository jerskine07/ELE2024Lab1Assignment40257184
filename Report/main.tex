\documentclass[a4paper,10pt,reqno]{amsart}

\usepackage[utf8]{inputenc}
\usepackage[foot]{amsaddr}
\usepackage{amsmath,amsfonts,amssymb,amsthm,mathrsfs,bm}
\usepackage[margin=0.95in]{geometry}
\usepackage{color}
\usepackage[dvipsnames]{xcolor}

\input{toc-config.tex}

\usepackage{mathtools,enumerate,mathrsfs,graphicx}
\usepackage{epstopdf}
\usepackage{hyperref}

\usepackage{latexsym}


\definecolor{CommentGreen}{rgb}{0.0,0.4,0.0}
\definecolor{Background}{rgb}{0.9,1.0,0.85}
\definecolor{lrow}{rgb}{0.914,0.918,0.922}
\definecolor{drow}{rgb}{0.725,0.745,0.769}

\usepackage{listings}
\usepackage{textcomp}
\lstloadlanguages{Matlab}%
\lstset{
    language=Matlab,
    upquote=true, frame=single,
    basicstyle=\small\ttfamily,
    backgroundcolor=\color{Background},
    keywordstyle=[1]\color{blue}\bfseries,
    keywordstyle=[2]\color{purple},
    keywordstyle=[3]\color{black}\bfseries,
    identifierstyle=,
    commentstyle=\usefont{T1}{pcr}{m}{sl}\color{CommentGreen}\small,
    stringstyle=\color{purple},
    showstringspaces=false, tabsize=5,
    morekeywords={properties,methods,classdef},
    morekeywords=[2]{handle},
    morecomment=[l][\color{blue}]{...},
    numbers=none, firstnumber=1,
    numberstyle=\tiny\color{blue},
    stepnumber=1, xleftmargin=10pt, xrightmargin=10pt
}

\numberwithin{equation}{section}
\synctex=1

\hypersetup{
    unicode=false, pdftoolbar=true,
    pdfmenubar=true, pdffitwindow=false, pdfstartview={FitH},
    pdftitle={ELE2024 Coursework}, pdfauthor={A. Author},
    pdfsubject={ELE2024 coursework}, pdfcreator={A. Author},
    pdfproducer={ELE2024}, pdfnewwindow=true,
    colorlinks=true, linkcolor=red,
    citecolor=blue, filecolor=magenta, urlcolor=cyan
}


% CUSTOM COMMANDS
\renewcommand{\Re}{\mathbf{re}}
\renewcommand{\Im}{\mathbf{im}}
\newcommand{\R}{\mathbb{R}}
\newcommand{\N}{\mathbb{N}}
\newcommand{\C}{\mathbb{C}}
\newcommand{\lap}{\mathscr{L}}
\newcommand{\dd}{\mathrm{d}}
\newcommand{\smallmat}[1]{\left[ \begin{smallmatrix}#1 \end{smallmatrix} \right]}

%opening
\title[ELE2024 Coursework]{LaTeX template for the ELE2024 coursework}

\author[A. Author]{An Author}
\author[B. Someone]{Bart Someone}

\address[A. Author and B. Someone]{. Email addresses: \href{mailto:some.author@qub.ac.uk}%
{some.author@qub.ac.uk} and
\href{mailto:a.student@qub.ac.uk}{a.student@qub.ac.uk}.}
\thanks{Some note goes here.
        Version 0.0.1. Last updated:~\today.}
\begin{document}

\maketitle


\section{Part A}

\subsection{Question Q1}\label{sec:q1}
You may format inline equations using the dollar sign
like that $x = 1 = \alpha$ and $y = x^2 - \sqrt{z}$.
Equations are like that:
\begin{align}
 \label{eq:lti_state_update}
 x_{k+1} = A x_k + Bu_k.
\end{align}
Here is an equation with the Laplace transform
\begin{equation}
    \lap \{e^{at}\} = \frac{1}{s-a},
\end{equation}
for all complex numbers $s\in\C$ with $\Re(s)>a$.
The inverse Laplace transform is denoted like this $\lap^{-1}$.




\subsection{Question Q2}
Refer to other sections as Section~\ref{sec:q1}. An example of a numbered list
\begin{enumerate}
 \item first item,
 \item second item.
\end{enumerate}
Links are \href{https://google.com}{like that}. We also have \textbf{boldface}, \textit{italics},
\emph{emphasised}, \texttt{truetype}, \textsc{Small Caps} and so on.
Format your MATLAB code as follows:
\begin{lstlisting}[language=matlab]
% My code:
f = @(x) sin(x);
y = f(0.1);
\end{lstlisting}

\subsection{More math} Denote the real numbers as $\R$ and the complex numbers
as $\C$. Example of a limit:
\begin{equation}
    z = \lim_{s\to0^+}\frac{s+1}{s^3+s^2-5s+9}.
\end{equation}
Another example
\begin{equation}
    \lim_{s\to\infty} \frac{s+1}{s^3+s^2-5s+9}.
\end{equation}
Example of an integral
\begin{equation}
    \int_0^\infty e^{-s\tau}f(\tau)\dd\tau.
\end{equation}
Three aligned equations
\begin{align}
    a =& 1,
    \\
    b =& 2,
    \\
    c =& 3.
\end{align}
Two aligned equations without equation numbers
\begin{align*}
    a =& 1,
    \\
    b =& 2.
\end{align*}
Mathematical derivations:
\begin{align}
    \frac{1}{2+3j} {}={}& \frac{2-3j}{(2+3j)(2-3j)}
    \notag\\
    {}={}& \frac{2-3j}{2^2 + 3^2}
    \notag\\
    {}={}& \frac{2-3j}{13}
    \notag\\
    {}={}&\frac{2}{13} - j\frac{3}{13}.
\end{align}
More mathematical derivations:
\begin{align*}
 & as + 4 + 2s = b + (8+a)s
\\
 \Leftrightarrow{}& (a+2)s + 4 = b + (8+a)s
\end{align*}
Boldface math: $\bm{x}$. Vectors:
\begin{equation}
\bm{x} =
    \begin{bmatrix}
    x_1
    \\
    x_2
    \\
    x_3
    \end{bmatrix}.
\end{equation}
Another example: According to Taylor's Theorem:
\begin{equation}
    \phi(x) \approx \phi(x_0) + \phi'(x_0)(x-x_0).
\end{equation}


% You can write comments like this
\begin{figure}
 \centering
 \includegraphics[width=0.6\linewidth]{figures/ode45_1.eps}
 \caption{You may of course include figures in your document. It is best to use vector format graphics such as EPS files.}
\end{figure}


\end{document}